% --------------------------------------------------------------------
% Warning
% change figure names for each screenshot
% NOTE for users:
% You need to   customize the colors used for code highlighting.
% Just edit the \definecolor commands or the keywordstyle settings.
%
% For example:
%   keywordstyle=[2]\color{purple}\bfseries   → change 'purple' to any color
%   keywordstyle=[3]\color{orange}\bfseries   → change 'orange' to another color
%
% Available colors: red, blue, green, orange, purple, teal, darkgray, etc.
% You can also define your own colors using \definecolor.
%
% This makes it easy to adapt the code style to your own preferences.

% Lines that are specific to my task and can be removed/changed:

%\lstdefinelanguage{JavaScript}{
%  keywords={typeof, module,new,model, true, false, function, return, null, switch, var, if, in, while, do, else, case, break, const, let},
%  keywordstyle=\color{blue}\bfseries,
%  ndkeywords={class, export, boolean, throw, implements, import, this},
%  ndkeywordstyle=\color{darkgray}\bfseries,
%  identifierstyle=\color{black},
%  sensitive=false,
%  comment=[l]{//},
%  morecomment=[s]{/*}{*/},
%  commentstyle=\color{teal}\ttfamily,
%  stringstyle=\color{red}\ttfamily,
%  morestring=[b]',
%  morestring=[b]",
%  % --- special group for try/catch ---
%  morekeywords=[2]{try,catch,exports},
%  morekeywords=[3]{mongoose, mongooseUrl}
%}
%
% If you don’t need special highlighting for `try/catch/exports` or `mongoose`,
% just delete these lines. The rest of the setup will still provide
% standard JavaScript syntax highlighting.
% --------------------------------------------------------------------


\documentclass[12pt,a4paper]{article}
\usepackage{graphicx} % Required for inserting images
\usepackage{amsmath}
\usepackage{listings}
\usepackage{xcolor}
\usepackage{geometry}
\geometry{margin=0.5in}
\usepackage{float}   % in preamble

\usepackage{verbatim}
\usepackage{caption}
\usepackage{color}

% Colors
\definecolor{lightgray}{rgb}{.9,.9,.9}
\definecolor{darkgray}{rgb}{.4,.4,.4}
\definecolor{purple}{rgb}{0.65, 0.12, 0.82}

% Define JavaScript with custom keyword groups
\lstdefinelanguage{JavaScript}{
  keywords={typeof, module,new,model, true, false, function, return, null, switch, var, if, in, while, do, else, case, break, const, let},
  keywordstyle=\color{blue}\bfseries,
  ndkeywords={class, export, boolean, throw, implements, import, this},
  ndkeywordstyle=\color{darkgray}\bfseries,
  identifierstyle=\color{black},
  sensitive=false,
  comment=[l]{//},
  morecomment=[s]{/*}{*/},
  commentstyle=\color{teal}\ttfamily,
  stringstyle=\color{red}\ttfamily,
  morestring=[b]',
  morestring=[b]",
  % --- special group for try/catch ---
  morekeywords=[2]{try,catch,exports},
  morekeywords=[3]{mongoose, mongooseUrl}
}

% Apply styles
\lstset{
   language=JavaScript,
   extendedchars=true,
   basicstyle=\footnotesize\ttfamily,
   showstringspaces=false,
   showspaces=false,
   tabsize=2,
   breaklines=true,
   showtabs=false,
   captionpos=b,
    backgroundcolor=\color{white},
    basicstyle=\ttfamily\small,
   keywordstyle=[2]\color{purple}\bfseries, % purple for try/catch
   keywordstyle=[3]\color{orange}\bfseries, % purple for try/catch
   frame=single
}

\title{Urban Wave Assignment}
\author{Dhruv Rana}
\date{28 August 2025}

\begin{document}
\maketitle

\section{MongoDB scripts for database creation and data insertion}
\subsection{Creation of database}
\begin{lstlisting}
const mongoose = require('mongoose');
const mongooseUrl = "mongodb://localhost:27017/BDA_Demo";
mongoose.connect(mongooseUrl).then(console.log("Connected to MongoDB")).catch(err => console.log(err));

\end{lstlisting}

\subsection{Schema Creation}
\subsubsection{User Schema}
\begin{lstlisting}
const mongoose=require('mongoose');
const {Schema} =require('mongoose');
const userSchema=new Schema({
    username:{
        type:String,
        unique:true,
        required:true
    },
    fullName:{
        type:String,
        required:true
    },
    email:{
        type:String,
        unique:true,
        required:true
    },
    age:{
        type:Number
    },
    bio:{
        type:String
    },
    interests:{
        type:[String],
    },
    location: {
    type: {
    type: String,
    enum: ['Point'],   
    required: true,
    default: 'Point'
  },
  coordinates: {
    type: [Number],    // [longitude, latitude]
    required: true,
    validate: {
      validator: function (value) {
        return value.length === 2; 
      },
      message: 'Coordinates must be [longitude, latitude]'
    }
  },
  city: {
    type: String,
    required: true,
    trim: true
  },
  state: {
    type: String,
    required: true,
    trim: true
  }
},
screenTime: {
    type: Number,
    required: true
},
followersCount:{
    type: Number,
    default: 0
},
followingCount:{
    type: Number,
    default: 0
},
isActive:{
    type: Boolean,
    default:true
    // required: true
},
createdAt:{
    type:Date,
    default:Date.now,
    required: true
},
lastSeen:{
    type:Date,
    default:Date.now,
    // required: true
}
})

module.exports=mongoose.model('User',userSchema);
\end{lstlisting}


\subsubsection{Post Schema}
\begin{lstlisting}
    const mongoose=require('mongoose');
const {Schema} =require('mongoose');

const postSchema=new Schema({
    userId:{
        type:mongoose.Schema.Types.ObjectId,
        ref:'User',
        required:true
    },
    type:{
        type:String,
        enum:['post','reel'],
        required:true
    },
    content:{
        type:String,
        required:true
    },
    caption:{
        type:String,
        required:true
    },
    hashtags:{
        type:[String]
    },
    
    location: {
    type: {
    type: String,
    enum: ['Point'],  
    required: true,
    default: 'Point'
  },
  coordinates: {
    type: [Number],    // [longitude, latitude]
    required: true,
    validate: {
      validator: function (value) {
        return value.length === 2; // Must have [lon, lat]
      },
      message: 'Coordinates must be [longitude, latitude]'
    }
  },
  city: {
    type: String,
    required: true,
    trim: true
  },
  state: {
    type: String,
    required: true,
    trim: true
  }
},
likes: {
    type: Number,
    default: 0
},
comments:{
    type: Number,
    default: 0
    
},
shares:{
    type: Number,
    default: 0
    
},
views:{
    type: Number,
    default: 0
},
createdAt:{
    type:Date,
    default:Date.now,
    required: true
},
})
module.exports=mongoose.model('Post',postSchema);
\end{lstlisting}

\subsubsection{Follower Schema}
\begin{lstlisting}
const mongoose=require('mongoose');
const {Schema}=mongoose;

const followerSchema=new Schema({
    userId:{
        type:mongoose.Schema.Types.ObjectId,
        ref:'User',
        required:true,
        unique:true
    },
    followerId:{
        type:[mongoose.Schema.Types.ObjectId],
        ref:'User',
        default:[],
        required:true
    },
    followingId:{
        type:[mongoose.Schema.Types.ObjectId],
        ref:'User',
        default:[],
        required:true
    },
    createdAt:{
        type:Date,
        default:Date.now,
        required:true
    }
});

module.exports=mongoose.model('Followers',followerSchema);    
\end{lstlisting}

\subsection{Data Insertion}
\subsubsection{Insertion of User}
\begin{lstlisting}
    expressRouter.get('/create_user', async (req, res) => {
    try {
        const userData = {
            username: "riya_fashion",
            fullName: "Riya Thakkar",
            email: "riyathakkar@gmail.com",
            age: 21,
            bio: "This is bio for Riya Thakkar",
            interests: ['fashion', 'beauty', 'lifestyle'],
            location: { type: "Point", coordinates: [73.2081, 22.3072] ,city:"Vadodara",state:"Gujarat"},
            screenTime: 250,
        };
        const user = await userSchema.create(userData);
        res.render("user.ejs", { user });
    } catch (error) {
        res.send("Error:" + error.message);
    }
});
\end{lstlisting}


\begin{figure}[h]
    \centering
    \includegraphics[width=0.55\textwidth]{user.png}
    \caption{User Insertion into MongoDB}
    \label{fig:sample}
\end{figure}


\subsubsection{Insertion of a post}
\begin{lstlisting}
    
expressRouter.get('/create_post', async (req, res) => {
    try {
        const post = await postSchema.create({
                userId:"68bed3c420a0aeac59c559cf",
                type:"reel", //change to post for uploading image 
                content:"coding",
                caption:"i am a new post by new priyansu",
                hashtags: ["#fashion","#travel","#coding"],
                location:{coordinates: [68.9685 ,22.2442], type: "Point", city:"Rajkot", state:"Gujarat"},
        });
        res.render("post.ejs",{post});
    } catch (error) {
        res.send(error.message);
    }
});
\end{lstlisting}


\begin{figure}[H]
    \centering
    \includegraphics[width=0.55\textwidth]{post.png}
    \caption{Post Insertion into MongoDB}
    \label{fig:sample}
\end{figure}

\subsubsection{Insertion of follow relationship}
\begin{lstlisting}
        const follower= await followerSchema.create({
            userId: "68af11b1ff8785cec12a60ae",
            followerId: "68af120be6f38faee252dcdd",
            followingId: "68af22aab8ac06908bc07bcc"
        });
        
\end{lstlisting}


\begin{figure}[H]
    \centering
    \includegraphics[width=0.6\textwidth]{follower.png}
    \caption{Post Insertion into MongoDB}
    \label{fig:sample}
\end{figure}


\section{Task Queries}
\subsection{CRUD Operations}

\subsubsection{Find all users from a specific city}
    \begin{lstlisting}
    
expressRouter.get('/fetchcity', async (req, res) => {
    try {
        const users = await userSchema.find({
            "location.city":{$eq:"Vadodara"}
        });
        res.render("all_users.ejs", { users });
    } catch (error) {
        res.send(error.message);
    }
});
    \end{lstlisting}
    

\begin{figure}[h]
    \centering
    \includegraphics[width=1.1\textwidth]{samecity.png}
    \caption{User from same city}
    \label{fig:sample}
\end{figure}

\subsubsection{Find users younger than 25 years}
    \begin{lstlisting}
        expressRouter.get('/fetchage', async (req, res) => {
    try {
        const users = await userSchema.find({
            age: { $lt: 25 }
        });
        res.render("all_users.ejs", { users });
    } catch (error) {
        res.send(error.message);
    }
});
    \end{lstlisting}
    \begin{figure}[H]
    \centering
    \includegraphics[scale=0.48]{agegt25.png}

    \caption{List of users with age less than 25}
    \label{fig:sample}
\end{figure}
    

\subsubsection{Find users with screen time greater than 150 minutes}
    \begin{lstlisting}
        expressRouter.get('/fetchst', async (req, res) => {
    try {
        const users = await userSchema.find({
            screenTime: { $gt: 150 }
        });
        res.render("all_users.ejs", { users });
    } catch (error) {
        res.send(error.message);
    }
});
    \end{lstlisting}
    \begin{figure}[H]
    \centering
    \includegraphics[scale=0.5]{screentime.png}

    \caption{Post Insertion into MongoDB}
    \label{fig:sample}
\end{figure}
    

\subsubsection{Update a user’s bio and interests}
    \begin{lstlisting}
expressRouter.get('/update', async (req, res) => {
    try {
        const users = await userSchema.findOneAndUpdate(
            { interests: 'coding' },
            {
                $set: { interests:['skating_changed','coding_changed','new_interest'] , bio:'This is updated bio for the user with interest in coding' },
                $currentDate: { lastModified: true }
            });
            res.send(users);
    } catch (error) {
        res.send(error.message);
    }
});
    \end{lstlisting}
    \begin{figure}[H]
    \centering
    \includegraphics[width=0.5\textwidth]{update.png}

    \caption{Updation of user interests and bio}
    \label{fig:sample}
\end{figure}
    

\subsubsection{Delete posts older than a specific date}
    \begin{lstlisting}
expressRouter.get('/deleteoldpost', async (req, res) => {
    try {
        console.log();
        const posts = await postSchema.findOne({ createdAt: { $lt:new Date(Date.now()-30*60*1000) } }); //post less than last 30 minutes
         await postSchema.deleteOne({ createdAt: { $lt:new Date(Date.now()-30*60*1000) } }); //post less last 30 minutes
        res.send( posts );
    } catch (error) {
        res.send(error.message);
    }
});
    \end{lstlisting}
    \begin{figure}[H]
    \centering
    \includegraphics[width=0.5\textwidth]{deletepost.png}

    \caption{Deletion of post}
    \label{fig:sample}
\end{figure}
    
\subsection{Data Filtering and Sorting}
\subsubsection{Find posts of type ”reel” with more than 100 likes}
\begin{lstlisting}
    expressRouter.get('/fetchlikes', async (req, res) => {
    try {
        const posts = await postSchema.find({ $and:[{type:'reel'},{likes:{$gt:100}}] });
        res.render("all_posts.ejs", { posts });
    } catch (error) {
        res.send(error.message);
    }
});
\end{lstlisting}
    \begin{figure}[H]
    \centering
    \includegraphics[width=0.9\textwidth]{like100.png}
    \caption{Post with more than 100 likes}
    \label{fig:sample}
\end{figure}

\subsubsection{Sort users by followers count (descending)}
\begin{lstlisting}
    
expressRouter.get('/sort', async (req, res) => {
    try {
        const users =await userSchema.find().sort({followersCount:-1});
            res.render("all_users.ejs", { users });
    } catch (error) {
        res.send(error.message);
    }
});

\end{lstlisting}
    \begin{figure}[H]
    \centering
    \includegraphics[scale=0.45]{sort.png}
    \caption{Users with decreasing order of follower}
    \label{fig:sample}
\end{figure}


\vspace{10in}
\subsubsection{Find users with specific interests (e.g., ”travel” and ”photography”)}
\begin{lstlisting}
expressRouter.get('/interests', async (req, res) => {
    try {
        const users = await userSchema.find(
            { interests: 'fashion' }
        );
        res.render("all_users.ejs", { users });
    } catch (error) {
        res.send(error.message);
    }
});

\end{lstlisting}
    \begin{figure}[H]
    \centering
    \includegraphics[scale=0.8]{interest.png}
    \caption{Users with specific interests}
    \label{fig:sample}
\end{figure}

\subsubsection{Get posts from the last 30 days”)}
\begin{lstlisting}

expressRouter.get('/fetcholdposts', async (req, res) => {
    try {
        console.log();
        const posts = await postSchema.find({ createdAt: { $gt:new Date(Date.now()-30*60*1000) } }); //post from last 30 minutes
        res.render("all_posts.ejs", { posts });
    } catch (error) {
        res.send(error.message);
    }
});

\end{lstlisting}
    \begin{figure}[H]
    \centering
    \includegraphics[scale=0.8]{oldpost_30min.png}
    \caption{Posts created 30 mins earlier}
    \label{fig:sample}
\end{figure}


\subsubsection{Get posts from the last 30 minutes}
\begin{lstlisting}
expressRouter.get('/active', async (req, res) => {
    try {
        const users =await userSchema.find({isActive:false});
            res.render("all_users.ejs", { users });
    } catch (error) {
        res.send(error.message);
    }
});

\end{lstlisting}
    \begin{figure}[H]
    \centering
    \includegraphics[scale=0.65]{active.png}
    \caption{Active users}
    \label{fig:sample}
\end{figure}

\section{Geospatial Queries}
\subsection{Location-Based Queries}
\subsubsection{Create a 2dsphere index on user locations}
\begin{lstlisting}
    db.users.createIndex({ location: "2dsphere" });
\end{lstlisting}
\begin{figure}[H]
    \centering
    \includegraphics[scale=0.5]{index.png}
    \caption{2d index for GeoLocation queries}
    \label{fig:sample}
\end{figure}

\subsubsection{Find users within 100km radius of Ahmedabad}
\begin{lstlisting}
    
expressRouter.get('/near', async (req, res) => {
    try {
        const users =await userSchema.find({
            location: {
                $near: {
                    $geometry: {
                        type: "Point", coordinates: [72.5714, 23.0225] //Ahmedabad
                    },
                    $maxDistance: 150000
                }
            }
        });
            res.render("all_users.ejs", { users });
    } catch (error) {
        res.send(error.message);
    }
});

\end{lstlisting}
\begin{figure}[H]
    \centering
    \includegraphics[scale=0.5]{near.png}
    \caption{User in 100km radius of Ahmedabad}
    \label{fig:sample}
\end{figure}

\subsubsection{Find the nearest user to a given coordinate}
\begin{lstlisting}
    
expressRouter.get('/nearcord', async (req, res) => {
    try {
        const users =await userSchema.findOne({
            location: {
                $near: {
                    $geometry: {
                        type: "Point", coordinates: [72.5714, 23.0225] //Ahmedabad
                    }
                }
            }
        });
            res.send(users);
    } catch (error) {
        res.send(error.message);
    }
});

\end{lstlisting}
\begin{figure}[H]
    \centering
    \includegraphics[scale=0.8]{nearcord.png}
    \caption{User near a coordinate}
    \label{fig:sample}
\end{figure}


\subsubsection{Get users from multiple cities using in operator}
\begin{lstlisting}
  
expressRouter.get('/incity', async (req, res) => {
    try {
        const users =await userSchema.find({
            "location.city": {$in: ['Ahmedabad','Surat','Vadodara']}
        });
            res.render("all_users.ejs",{users});
    } catch (error) {
        res.send(error.message);
    }
});


\end{lstlisting}
\begin{figure}[H]
    \centering
    \includegraphics[scale=0.65]{incity.png}
    \caption{Users in given cities}
    \label{fig:sample}
\end{figure}


\subsubsection{Calculate distance between two users}
\begin{lstlisting}
  


\end{lstlisting}
\begin{figure}[H]
    \centering
    \includegraphics[scale=0.65]{distance.png}
    \caption{Distance between two users}
    \label{fig:sample}
\end{figure}




\end{document}



